\section{Challenges}
\label{sec:implementation:challenges}

Based on the GPU architecture and programming principles discussed in Chapter \ref{sec:background}, translating the Cuckoo filter algorithm into a performant parallel implementation presents several specific technical hurdles. The following challenges directly influenced the design and optimization of the kernels detailed in this chapter:

\begin{itemize}
  \item \textbf{Managing Concurrency and Race Conditions}: A naive parallel implementation where thousands of GPU threads attempt to read and write to the same buckets simultaneously would lead to race conditions and data corruption. Designing an efficient, lock-free mechanism using atomic operations to handle these concurrent memory accesses is one of the primary challenges.

  \item \textbf{Parallelizing the Eviction Path}: During an insertion, if both candidate buckets are full, an existing item must be evicted and reinserted into its alternate location. This eviction process can cascade, leading to multiple evictions. Parallelizing this inherently path-dependent process is complex, as it requires careful coordination to ensure that threads do not interfere with each other's operations or create deadlocks.

  \item \textbf{Optimizing for Coalesced Memory Access}: GPUs are highly sensitive to memory access patterns. Coalesced memory accesses, where threads in a warp access contiguous memory locations, are crucial for achieving high throughput. The random-access nature of Cuckoo filter operations can lead to scattered, uncoalesced memory accesses, which severely degrade performance. Developing strategies like data sorting to improve memory locality is essential.

  \item \textbf{Balancing Occupancy and Register Pressure}: Achieving a high and balanced occupancy across the filter is key to its space efficiency. However, complex insertion logic can increase the number of registers used per thread. High register usage can limit the number of active warps on a Streaming Multiprocessor (SM), thereby reducing occupancy and the hardware's ability to hide memory latency.
\end{itemize}
