\definecolor{hashRed}{RGB}{210, 40, 40}
\definecolor{hashGreen}{RGB}{0, 140, 60}
\definecolor{hashBlue}{RGB}{40, 90, 200}
\definecolor{bitSetFill}{RGB}{235, 242, 250}
\definecolor{textDark}{RGB}{40, 40, 40}

\begin{figure}[ht!]
  \centering
  \begin{tikzpicture}[
      every node/.style={text=textDark},
      cell/.style={
        draw=gray!80,
        line width=0.6pt,
        minimum size=0.85cm,
        inner sep=0pt,
        font=\large,
        fill=white,
      },
      cellset/.style={
        cell,
        fill=bitSetFill,
      },
      blocklabel/.style={
        font=\normalsize,
        text=textDark,
      },
      stem/.style={line width=0.7pt},
      arr/.style={
        line width=0.7pt,
        -{Stealth[length=4pt, width=3.5pt]}
      },
    ]

    \def\cw{0.85}   % cell width
    \def\hc{0.425}  % half cell size
    \def\bg{0.6}    % gap between blocks

    % Block 0
    \node[cell]    (c0) at (0,        0) {0};
    \node[cellset] (c1) at ({1*\cw},  0) {1};
    \node[cell]    (c2) at ({2*\cw},  0) {0};
    \node[cellset] (c3) at ({3*\cw},  0) {1};

    % Block 1
    \pgfmathsetmacro{\boff}{4*\cw + \bg}
    \node[cellset] (c4) at ({\boff + 0*\cw}, 0) {1};
    \node[cell]    (c5) at ({\boff + 1*\cw}, 0) {0};
    \node[cellset] (c6) at ({\boff + 2*\cw}, 0) {1};
    \node[cellset] (c7) at ({\boff + 3*\cw}, 0) {1};

    % Block labels
    \node[blocklabel] at ({1.5*\cw}, -0.9) {Block 0};
    \node[blocklabel] at ({\boff + 1.5*\cw}, -0.9) {Block 1};

    % Set label
    % Center between the two blocks
    \pgfmathsetmacro{\midx}{(\boff + 3*\cw) / 2}
    \node[font=\large] (setlabel) at (\midx, 3.6) {$\{x,\, y,\, z\}$};

    % Origin points below the label, one per element
    \coordinate (ox) at ($(setlabel.south) + (-0.55, -0.15)$);
    \coordinate (oy) at ($(setlabel.south) + ( 0.0,  -0.15)$);
    \coordinate (oz) at ($(setlabel.south) + ( 0.55, -0.15)$);

    % Rail heights
    \coordinate (rR) at (0, 2.3);    % x — highest
    \coordinate (rB) at (0, 1.7);    % z — middle
    \coordinate (rG) at (0, 1.1);    % y — lowest
    \coordinate (rW) at (0, -1.4);   % w — below array

    % x → Block 1: cells 4, 6
    \draw[hashRed, stem] (ox) -- (ox |- rR);
    \draw[hashRed, arr]  (ox |- rR) -| ({\boff + 0*\cw - 0.12}, \hc);
    \draw[hashRed, arr]  (ox |- rR) -| ({\boff + 2*\cw}, \hc);

    % z → Block 1: cells 4, 7
    \draw[hashBlue, stem] (oz) -- (oz |- rB);
    \draw[hashBlue, arr]  (oz |- rB) -| ({\boff + 0*\cw + 0.12}, \hc);
    \draw[hashBlue, arr]  (oz |- rB) -| ({\boff + 3*\cw}, \hc);

    % y → Block 0: cells 1, 3
    \draw[hashGreen, stem] (oy) -- (oy |- rG);
    \draw[hashGreen, arr]  (oy |- rG) -| ({1*\cw}, \hc);
    \draw[hashGreen, arr]  (oy |- rG) -| ({3*\cw}, \hc);

    % w → Block 0: cells 0, 1
    \node[font=\large] (wlabel) at ({1.5*\cw}, -2.1) {$w$};
    \draw[textDark, stem] (wlabel.north) -- (wlabel.north |- rW);
    \draw[textDark, arr]  (wlabel.north |- rW) -| ({0*\cw}, -\hc);
    \draw[textDark, arr]  (wlabel.north |- rW) -| ({3*\cw}, -\hc);

  \end{tikzpicture}
  \caption{An illustration of a Blocked Bloom filter.
    Each item is first mapped to a single block (e.g., item $y$ maps to Block 0, while $x$ and $z$ map to Block 1).
    The $k$ hash functions then operate only within that selected block, improving memory locality.
  A query for item $w$, which maps to Block 0, returns a definitive "no" as one of its target bits is 0.}
  \label{fig:blocked-bloom-filter}
\end{figure}
