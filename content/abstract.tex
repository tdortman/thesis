% !TEX root = ../my-thesis.tex
%
\IfLanguageName{english}{\pdfbookmark[0]{Abstract}{Abstract}}{\pdfbookmark[0]{Zusammenfassung}{Zusammenfassung}}
\IfLanguageName{english}{\addchap*{Abstract}}{\addchap*{Zusammenfassung}}
\label{sec:abstract}

Approximate Membership Query (AMQ) data structures are critical for high-throughput systems, yet the industry-standard Bloom filter cannot support item deletion. This thesis presents the design and implementation of a high-performance, GPU-accelerated Cuckoo filter, provided as a robust header-only library. By leveraging efficient algorithms built on atomic operations, the implementation achieves massive parallelism without the need for explicit locking.

A comprehensive evaluation across GDDR7 and HBM3 architectures demonstrates that this Cuckoo filter significantly outperforms state-of-the-art dynamic alternatives. It achieves insertion and deletion throughputs orders of magnitude higher than the GPU Quotient filter (GQF) and the Two-Choice filter (TCF). Crucially, the results reveal superior architectural scalability: while TCF and GQF performance stagnates on high-end hardware due to internal latency bottlenecks, the Cuckoo filter scales linearly with global memory bandwidth. This characteristic positions it as the optimal solution for next-generation, bandwidth-rich GPU architectures.

\IfLanguageName{english}{
  \vspace*{20mm}

  {\usekomafont{chapter}Abstract (German)}
  \label{sec:abstract-diff}
  % Set german quote style and language system.
  \setquotestyle{german}
  \selectlanguage{ngerman}

  Datenstrukturen für Approximate Membership Queries (AMQ) sind in Hochdurchsatzsystemen unverzichtbar, doch dem weit verbreiteten Bloom-Filter fehlt die Unterstützung für Löschoperationen. Diese Arbeit stellt einen hochperformanten, GPU-beschleunigten Cuckoo-Filter vor, der als robuste Header-only-Bibliothek implementiert wurde. Durch effiziente Algorithmen auf Basis atomarer Operationen ermöglicht die Implementierung massive Parallelität ohne explizites Locking.

  Evaluierungen auf GDDR7- und HBM3-Systemen zeigen, dass der Filter dynamische Alternativen deutlich übertrifft. Er erzielt Einfüge- und Löschraten, die um Größenordnungen über denen des GPU Quotient Filters (GQF) und des Two-Choice Filters (TCF) liegen. Ein entscheidender Vorteil ist die Skalierbarkeit: Während die Leistung des TCF und GQF auf moderner Hardware latenzbedingt stagniert, skaliert der Cuckoo-Filter linear mit der globalen Speicherbandbreite. Dies macht ihn zur führenden Lösung für zukünftige GPU-Architekturen mit hoher Speicherbandbreite.

  % reset to english
  \selectlanguage{english}
  \setquotestyle{english}

}
{}
