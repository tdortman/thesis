\section{Use Cases}
\label{sec:intro:use-cases}

The demand for high-speed, dynamic set membership testing shows up across numerous domains.
While CPU-based dynamic filters exist, many modern applications generate data at a rate that creates a performance bottleneck, motivating the need for a massively parallel, GPU-accelerated solution.
Some of the areas that could benefit from such a filter are as follows:

\begin{itemize}
  \item \textbf{Network Security}: Threat intelligence feeds for malicious IPs, URLs, and malware signatures are updated continuously, requiring a filter that supports both insertions and deletions.
    In high-speed networks (100Gbps+), a CPU can be overwhelmed by the sheer volume of packets per second.
    A GPU-accelerated filter can process large batches of packet headers or identifiers in parallel, enabling line-rate inspection against dynamic blacklists in a way that is simply infeasible for CPU-based solutions. \cite{grashofer2018towards, packet-classification, cuckooswitch}

  \item \textbf{Caching Systems}: Caches in Content Delivery Networks (CDNs) and HTTP reverse proxies experience constant churn as items are added and evicted.
    A dynamic filter is essential to prevent expensive disk or network lookups for non-existent objects.
    When the request rate is in the millions per second, the CPU can offload these cache-presence checks to a GPU, processing them in large batches to free up cycles for handling the actual data I/O. \cite{chang2008bigtable, counting-bloom}

  \item \textbf{Databases and Distributed Systems}: Database systems often use filters to avoid expensive disk lookups for non-existent keys.
    In a distributed setting, a GPU-accelerated filter could serve as a high-performance shared resource that tracks the existence of records across multiple nodes, reducing network latency and improving overall query performance. \cite{dayan2021chucky, mosharraf2022improving, ren2017slimdb, yao2023mdcf}

  \item \textbf{Bioinformatics}: Genomics and proteomics research involves searching for patterns or sequences within massive biological datasets.
    A GPU-accelerated Cuckoo filter could be used to rapidly pre-screen for the presence of specific markers before launching more computationally intensive analyses, significantly speeding up the research pipeline. \cite{gaia2019ngsreadstreatment, melsted2011efficient}
\end{itemize}