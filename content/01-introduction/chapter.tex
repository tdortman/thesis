% !TEX root = ../../my-thesis.tex
\chapter{Introduction}
\label{sec:intro}

In the era of big data, the ability to perform set membership queries is a fundamental requirement for applications ranging from network traffic analysis \cite{bloom-network} to large-scale distributed systems \cite{bloom-distributed-systems}. Determining whether an element belongs to a set is a frequent and performance-critical operation. While exact data structures provide definitive answers, their memory footprint and computational overhead can make them impractical for massive datasets. This has led to the widespread adoption of probabilistic data structures, which trade a small, manageable false positive probability for significant gains in space and time efficiency.

For years, the Bloom filter has been the standard probabilistic data structure for approximate set membership. Its primary limitation, however, is the inability to delete elements, making it unsuitable for dynamic datasets. While there are variations that do support deletion, they incur prohibitive space overheads that often negate their practical viability. Due to this the Cuckoo filter has emerged as a powerful alternative, offering native support for deletions and often superior space efficiency, particularly at low false positive rates.

\section{Problem Statement}
\label{sec:intro:problem}

Despite these advantages, the performance of Cuckoo filters on CPUs can quickly become a bottleneck in high-throughput environments. The sequential nature of the insertion algorithm, which may involve displacing multiple existing items, severely limits scalability.

This performance gap motivates the exploration of massively parallel hardware. However, porting a Cuckoo filter to a GPU is not a straightforward translation, as the algorithm's reliance on sequential eviction chains and random memory accesses conflicts with the GPU's desire for massive parallelism and structured, contiguous memory access patterns. This thesis addresses these architectural mismatches by designing, implementing, and evaluating a GPU-accelerated Cuckoo filter. The main objective is to leverage the massive parallelism of modern GPUs to handle insertions, lookups, and deletions, thereby achieving a significant performance leap over existing CPU-based implementations.

\section{Use Cases}
\label{sec:intro:use-cases}

The demand for high-speed, dynamic set membership testing shows up across numerous domains.
While CPU-based dynamic filters exist, many modern applications generate data at a rate that creates a performance bottleneck, motivating the need for a massively parallel, GPU-accelerated solution.
Some of the areas that could benefit from such a filter are as follows:

\begin{itemize}
  \item \textbf{Network Security}: Threat intelligence feeds for malicious IPs, URLs, and malware signatures are updated continuously, requiring a filter that supports both insertions and deletions.
    In high-speed networks (100Gbps+), a CPU can be overwhelmed by the sheer volume of packets per second.
    A GPU-accelerated filter can process large batches of packet headers or identifiers in parallel, enabling line-rate inspection against dynamic blacklists in a way that is simply infeasible for CPU-based solutions. \cite{grashofer2018towards, packet-classification, cuckooswitch}

  \item \textbf{Caching Systems}: Caches in Content Delivery Networks (CDNs) and HTTP reverse proxies experience constant churn as items are added and evicted.
    A dynamic filter is essential to prevent expensive disk or network lookups for non-existent objects.
    When the request rate is in the millions per second, the CPU can offload these cache-presence checks to a GPU, processing them in large batches to free up cycles for handling the actual data I/O. \cite{chang2008bigtable, counting-bloom}

  \item \textbf{Databases and Distributed Systems}: Database systems often use filters to avoid expensive disk lookups for non-existent keys.
    In a distributed setting, a GPU-accelerated filter could serve as a high-performance shared resource that tracks the existence of records across multiple nodes, reducing network latency and improving overall query performance. \cite{dayan2021chucky, mosharraf2022improving, ren2017slimdb, yao2023mdcf}

  \item \textbf{Bioinformatics}: Genomics and proteomics research involves searching for patterns or sequences within massive biological datasets.
    A GPU-accelerated Cuckoo filter could be used to rapidly pre-screen for the presence of specific markers before launching more computationally intensive analyses, significantly speeding up the research pipeline. \cite{gaia2019ngsreadstreatment, melsted2011efficient}
\end{itemize}
\section{Contributions}
\label{sec:intro:contributions}

This thesis presents a comprehensive study on accelerating probabilistic data structures using GPUs. The main contributions of this work are as follows:

\begin{itemize}
  \item \textbf{A novel GPU-Accelerated Cuckoo Filter Design}: The design and implementation of a parallel Cuckoo filter supporting insertion, lookup, and deletion are presented. The implementation introduces a lock-free, atomic-based mechanism to handle the "cuckoo" eviction process safely within a massively parallel environment, addressing the challenge of managing race conditions without sacrificing throughput.

  \item \textbf{Advanced Optimization Techniques}: Several optimization strategies are explored and evaluated to maximize occupancy and memory bandwidth. These include a sorted-insertion algorithm to improve memory locality and a modified eviction strategy designed to reduce eviction chain lengths at high load factors.

  \item \textbf{System-Level Integration Extensions}: Moving beyond a standalone kernel, two extensions are contributed to facilitate real-world adoption. First, an Inter-Process Communication (IPC) wrapper is developed to enable zero-copy, low-latency sharing of the filter between processes. Second, a multi-GPU implementation is provided that transparently partitions data across multiple devices using NCCL, allowing the filter to scale beyond the memory limits of a single card.

  \item \textbf{Comprehensive Evaluation}: A rigorous analysis is conducted comparing the GPU Cuckoo filter against CPU baselines and other GPU-accelerated filters. The results demonstrate that the implementation achieves competitive throughput with the static Blocked Bloom filter while far surpassing the other tested dynamic filters.
\end{itemize}

\section{Thesis Structure}
\label{sec:intro:structure}

\textbf{Chapter \ref{sec:background}} \\[0.2em]
This chapter establishes the foundation for the thesis. It goes over the fundamental concepts of approximate membership query structures, tracing the evolution from the classic Bloom filter to the Cuckoo filter. Additionally, it examines alternative modern data structures, such as the Quotient Filter and the Two-Choice Filter, to contextualize the research landscape. Subsequently, the chapter introduces the architectural principles of GPU computing and the CUDA programming model, highlighting the specific hardware constraints that influence parallel algorithm design.

\textbf{Chapter \ref{sec:implementation}} \\[0.2em]
This chapter details the design and implementation of the high-performance GPU Cuckoo filter library. It documents the parallel algorithms developed for insertion, lookup, and deletion, explaining how lock-free atomic operations are utilized to manage parallel accesses. Furthermore, it describes some advanced optimizations for memory locality and system-level extensions, including an Inter-Process Communication wrapper and a multi-GPU sharding mechanism.

\textbf{Chapter \ref{sec:evaluation}} \\[0.2em]
This chapter presents a comprehensive empirical analysis of the implemented data structure. The filter is benchmarked against state-of-the-art CPU and GPU baselines across diverse hardware architectures (GDDR7 and HBM3). Beyond raw throughput, the evaluation scrutinizes architectural scaling behaviours, analysing the impact of the memory hierarchy (L2 vs. DRAM), hardware utilization, and cache efficiency. Furthermore, it quantifies the impact of specific algorithmic optimizations, including eviction policies, bucket sizing, and sorted insertion, and validates the system's real-world applicability through multi-GPU scalability tests and a genomic $k$-mer indexing benchmark.

\textbf{Chapter \ref{sec:conclusion}} \\[0.2em]
The final chapter highlights the key findings of the research, summarizing the contributions made to the field of parallel probabilistic data structures. It critically assesses the limitations of the current implementation and outlines potential avenues for future research and optimization.
