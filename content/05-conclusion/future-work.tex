\section{Future Work}
\label{sec:conclusion:future-work}

The work presented in this thesis opens several avenues for future research and optimisation:

\begin{itemize}
  \item \textbf{Asynchronous APIs}: The current IPC mechanism is blocking.
    Future iterations could implement an asynchronous command queue, similar to \texttt{io\_uring}, to allow clients to submit batches of requests without stalling, further maximising GPU occupancy.

  \item \textbf{Variable-Length Fingerprints}: Investigating methods to support variable-length fingerprints within the fixed-bucket structure could allow users to fine-tune the space-accuracy trade-off without the coarseness of jumping from 16-bit to 32-bit tags.

  \item \textbf{Integration into High-Throughput Systems}: While micro-benchmarks demonstrate raw speed, the ultimate validation would be integrating the library into real-world systems characterised by high churn.
    Deploying the filter within network intrusion detection systems, streaming analytics pipelines, or GPU-accelerated database engines would provide valuable insight into its impact on end-to-end application latency and throughput.

  \item \textbf{Robustness Against Churn}: Future work could explore hybrid architectures, such as attaching a small "overflow stash" or secondary table, to catch items that fail the eviction chain.
    This would significantly improve stability and reliability for non-terminating, dynamic workloads.
\end{itemize}