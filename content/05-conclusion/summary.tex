\section{Summary of Contributions}
\label{sec:conclusion:summary}

This thesis presented the design, implementation, and evaluation of a high-performance Cuckoo filter accelerated by Graphics Processing Units. The primary motivation was to address the performance gap between static probabilistic data structures, like the Bloom filter, and dynamic ones needed for high-churn applications.

The key contributions of this work are:

\begin{itemize}
  \item \textbf{High-Performance CUDA Library}: A robust, header-only CUDA library was developed. This library encapsulates the parallel algorithms for insertion, lookup, and deletion, abstracting the complexity of GPU memory management and atomic synchronization. It provides a flexible, template-based C++ interface that allows users to configure critical parameters such as fingerprint size and hashing strategy at compile time for maximum efficiency.

  \item \textbf{Architectural Optimization}: The Cuckoo filter was tuned for GPU hardware, identifying a bucket size of 16 as the optimal trade-off between memory bandwidth utilization and eviction complexity.

  \item \textbf{System Extensions}: To facilitate real-world adoption, the filter was extended with an IPC wrapper for zero-copy sharing and a multi-GPU sharding mechanism to scale beyond the memory limits of a single device.

  \item \textbf{Rigorous Evaluation}: A comprehensive analysis was conducted across different memory technologies (GDDR7 and HBM3). The results demonstrate that the implementation successfully bridges the gap between static and dynamic filters, offering update capabilities orders of magnitude faster than existing alternatives like the Quotient Filter while maintaining query throughput competitive with the Blocked Bloom filter.
\end{itemize}